\chapter{Algebra and co-algebras}

 Recall that an algebra $\mathcal{A}$ is given by a multiplication map $m\colon \mathcal{A}\otimes\mathcal{A}\to\mathcal{A}$ and a unit map $\eta\colon \CC\to\mathcal{A}$, with associativity $m(m\otimes\id)=m(\id\otimes\,m)$ and the property $m(\eta\otimes\id)=\id=m(\id\otimes\,\eta)$. We say $(\mathcal{A},m,\eta)$ is an algebra when those properties are satisfied.

 Also recall that we say $\mathcal{A}$ is a \textit{co-algebra} when it has a co-multiplication map $\mu\colon\mathcal{A}\to\mathcal{A}\otimes\mathcal{A}$ and a co-unit map $\varpi\colon\mathcal{A}\to\CC$ such that co-associativity is satisfied, i.e., $(\mu\otimes\id)\mu=(\id\otimes\,\mu)\mu$, and the property $(\varpi\otimes\id)\mu=\id=(\id\otimes\,\varpi)\mu$ is satisfied.
 We say $(\mathcal{A},\mu,\varpi)$ is a co-algebra when those properties are satisfied.

 \begin{theorem}[the Frobenius equations]\label{Frobenius_equations}\lean{lTensor_mul_comp_rTensor_comul_of}\leanok
  Given an algebra and co-algebra $(\mathcal{A},m,\eta,\mu,\varpi)$, then if
  \[(\id\otimes\,m)(\mu\otimes\id)=(m\otimes\id)(\id\otimes\,\mu),\]
  then we get the following equations (the Frobenius equations),
  \[(\id\otimes\,m)(\mu\otimes\id)=\mu m=(m\otimes\id)(\id\otimes\,\mu).\]
 \end{theorem}
 \begin{proof}\leanok
  \begin{align*}
   (m\otimes\id)(\id\otimes\,\mu) &= (m\otimes\id)((\varpi\otimes\id)\mu\otimes\mu) &\text{by (counit.id)}\\
   &= (m\otimes\id)(\varpi\otimes\id^{\otimes3})\mu^{\otimes2}\\
   &= (m(\varpi\otimes\id^{\otimes2})\otimes\id)\mu^{\otimes2}\\
   &= ((\id_\CC\otimes\,m)(\varpi\otimes\id^{\otimes2})\otimes\id)\mu^{\otimes2}\\
   &= ((\varpi\otimes\id)(\id\otimes\,m)\otimes\id)\mu^{\otimes2}\\
   &= (\varpi\otimes\id^{\otimes2})(\id\otimes\,m\otimes\id)(\mu\otimes\id^{\otimes2})(\id\otimes\,\mu)\\
   &= (\varpi\otimes\id^{\otimes2})((\id\otimes\,m)(\mu\otimes\id)\otimes\id)(\id\otimes\,\mu)\\
   &= (\varpi\otimes\id^{\otimes2})((m\otimes\id)(\id\otimes\,\mu)\otimes\id)(\id\otimes\,\mu) &\text{by hyp}\\
   &= (\varpi\otimes\id^{\otimes2})(m\otimes\id^{\otimes2})(\id\otimes\,(\mu\otimes\id)\mu)\\
   &= (\varpi m\otimes\id^{\otimes2})(\id\otimes\,(\id\otimes\mu)\mu) &\text{by (comul.assoc)}\\
   &= (\varpi\otimes\id^{\otimes2})(m\otimes\id^{\otimes2})(\id^{\otimes2}\otimes\mu)(\id\otimes\,\mu)\\
   &= (\varpi\otimes\id^{\otimes2})(\id\otimes\,\mu)(m\otimes\id)(\id\otimes\,\mu)\\
   &= (\varpi\otimes\id^{\otimes2})(\id\otimes\,\mu)(\id\otimes\,m)(\mu\otimes\id) &\text{by hyp}\\
   &= (\varpi\otimes\id^{\otimes2})(\id\otimes\,\mu m)(\mu\otimes\id)\\
   &= (\id_\CC\otimes\,\mu m)(\varpi\otimes\id^{\otimes2})(\mu\otimes\id)\\
   &= \mu m((\varpi\otimes\id)\mu\otimes\id)\\
   &= \mu m(\id\otimes\,\id) &\text{by (counit.id)}\\
   &= \mu m.
  \end{align*}
 \end{proof}

 \begin{definition}\label{FrobeniusAlgebra}\lean{FrobeniusAlgebra}\leanok
  We say an algebra and co-algebra $(\mathcal{A},m,\eta,\mu,\varpi)$ is a \textit{Frobenius algebra} when it satisfies the Frobenius equation condition,
  \[(m\otimes\id)(\id\otimes\,\mu)=(\id\otimes\,m)(\mu\otimes\id).\]
 \end{definition}

 \begin{corollary}\label{rTensor_mul_comp_lTensor_comul_unit_eq_comul}
  \uses{FrobeniusAlgebra, Frobenius_equations}
  \lean{FrobeniusAlgebra.rTensor_mul_comp_lTensor_comul_unit_eq_comul}\leanok
  Given a Frobenius algebra $(\mathcal{A},m,\eta,\mu,\varpi)$, we get,
  \[(m\otimes\id)(\id\otimes\,\mu\eta)=\mu.\]
 \end{corollary}
 \begin{proof}\leanok
  By Theorem \ref{Frobenius_equations} and definition.
 \end{proof}

 \begin{corollary}\label{lTensor_mul_comp_rTensor_comul_unit}
  \uses{FrobeniusAlgebra, Frobenius_equations}
  \lean{FrobeniusAlgebra.lTensor_mul_comp_rTensor_comul_unit}
  Given a Frobenius algebra $(\mathcal{A},m,\eta,\mu,\varpi)$, we get,
  \[(\id\otimes\,m)(\mu\unit\otimes\id)=\mu.\]
 \end{corollary}
 \begin{proof}\leanok
  By Theorem \ref{Frobenius_equations} and definition.
 \end{proof}

 \begin{corollary}\label{lTensor_counit_mul_comp_rTensor_comul}
  \uses{lTensor_mul_comp_rTensor_comul_unit}
  \lean{FrobeniusAlgebra.lTensor_counit_mul_comp_rTensor_comul}\leanok
  Given a Frobenius algebra $(\mathcal{A},m,\eta,\mu,\varpi)$, we get,
  \[(\id\otimes\,\varpi m)(\mu\otimes\id)=m.\]
 \end{corollary}
 \begin{proof}\leanok
  Use Corollary \ref{lTensor_mul_comp_rTensor_comul_unit}.
 \end{proof}
 \begin{corollary}\label{rTensor_counit_mul_comp_lTensor_comul}
  \uses{rTensor_mul_comp_lTensor_comul_unit_eq_comul}
  \lean{FrobeniusAlgebra.rTensor_counit_mul_comp_lTensor_comul}\leanok
  Given a Frobenius algebra $(\mathcal{A},m,\eta,\mu,\varpi)$, we get,
  \[(\varpi m\otimes\id)(\id\otimes\,\mu)=m.\]
 \end{corollary}
 \begin{proof}\leanok
  Use Corollary \ref{rTensor_mul_comp_lTensor_comul_unit_eq_comul}.
 \end{proof}

 The following two results are known as the ``snake equations''.
 \begin{corollary}\label{rTensor_counit_mul_comp_lTensor_comul_unit}
  \uses{rTensor_counit_mul_comp_lTensor_comul}
  \lean{FrobeniusAlgebra.rTensor_counit_mul_comp_lTensor_comul_unit}\leanok
  Given a Frobenius algebra $(\mathcal{A},m,\eta,\mu,\varpi)$, we get,
  \[(\varpi m\otimes\id)(\id\otimes\,\mu\eta)=\id.\]
 \end{corollary}
 \begin{proof}\leanok
  Use Corollary \ref{rTensor_counit_mul_comp_lTensor_comul}.
 \end{proof}
 \begin{corollary}\label{lTensor_counit_mul_comp_rTensor_comul_unit}
  \uses{lTensor_counit_mul_comp_rTensor_comul}
  \lean{FrobeniusAlgebra.lTensor_counit_mul_comp_rTensor_comul_unit}\leanok
  Given a Frobenius algebra $(\mathcal{A},m,\eta,\mu,\varpi)$, we get,
  \[(\id\otimes\,\varpi m\otimes)(\mu\eta\otimes\id)=\id.\]
 \end{corollary}
 \begin{proof}\leanok
  Use Corollary \ref{lTensor_counit_mul_comp_rTensor_comul}.
 \end{proof}

\section{Finite-dimensional algebras}
 
 \begin{proposition}\label{Coalgebra.ofFiniteDimensionalHilbertAlgebra}
  \lean{Coalgebra.ofFiniteDimensionalHilbertAlgebra}\leanok
  Let $(\mathcal{A},m,\eta)$ be a finite-dimensional algebra and Hilbert space. Then we can form a co-algebra by letting $m^*$ be the co-multiplication and $\eta^*$ be the co-unit.
 \end{proposition}
 \begin{proof}\leanok {\ }
  \begin{description}
   \item[comul\_assoc:]
    We have the following equivalences,
    \begin{align*}
      (m^*\otimes\id)m^*=(\id\otimes\,m^*)m^* &\LRa {(m(m\otimes\id))}^*={(m(\id\otimes\,m)}^*\\
      &\LRa m(m\otimes\id)=m(\id\otimes\,m),
    \end{align*}
    which is true since $\mathcal{A}$ is an algebra.
   \item[counit\_comul\_id:]
    Similarly, taking adjoints, we get
    \begin{align*}
     &\, (\eta^*\otimes\id)m^*=\id=(\id\otimes\,\eta^*)m^*
     \,\LRa\, m(\eta\otimes\id)=\id=m(\id\otimes\,\eta),
    \end{align*}
    which is true since $\mathcal{A}$ is an algebra.
  \end{description}
 \end{proof}
 
 \begin{lemma}\label{counit_eq_bra_one}
  \uses{Coalgebra.ofFiniteDimensionalHilbertAlgebra}
  \lean{Coalgebra.counit_eq_bra_one}\leanok
  Let $(\mathcal{A},m,\eta)$ be a finite-dimensional algebra and Hilbert space. Then the counit is exactly $\bra{1}$.
 \end{lemma}
 \begin{proof}\uses{unit_adjoint_eq_bra_one}\leanok
  By definition and Lemma \ref{unit_adjoint_eq_bra_one}.
 \end{proof}

 \begin{proposition}\label{isAlgHom_iff_adjoint_isCoalgHom}
  \uses{Coalgebra.ofFiniteDimensionalHilbertAlgebra}
  \lean{LinearMap.isAlgHom_iff_adjoint_isCoalgHom}\leanok
  Let $(\mathcal{A}_1,m_1,\eta_1),(\mathcal{A}_2,m_2,\eta_2)$ be finite-dimensional algebras and Hilbert spaces, and let $f\colon\mathcal{A}_1\to\mathcal{A}_2$ be a linear map. Then $f$ is an algebra homomorphism if and only if $f^*$ is a co-algebra homomorphism.
 \end{proposition}
 \begin{proof}\leanok
  \begin{align*}
   f^*\text{ is a co-algebra hom} &\LRa (f^*\otimes f^*)\circ m_2^*=m_1^*\circ f^* \text{ and }\eta_1^*\circ f^*=\eta_2^*\\
   &\LRa m_2\circ(f\otimes f)=f\circ m_1\text{ and }f\circ\eta_1=\eta_2\\
   &\LRa f\text{ is an algebra hom}.
  \end{align*}
 \end{proof}

 \begin{proposition}\label{FiniteDimensionalCoAlgebra_isFrobeniusAlgebra_of}
  \uses{FrobeniusAlgebra}
  \lean{FiniteDimensionalCoAlgebra_isFrobeniusAlgebra_of}\leanok
  For any finite-dimensional algebra $(B,m,\eta)$ that is also a Hilbert space, if $\ip{xy}{z}=\ip{y}{x^*z}$ for all $x,y,z\in B$, then $(B,m,\eta)$ is a Frobenius algebra.
 \end{proposition}
 \begin{proof}\uses{Coalgebra.ofFiniteDimensionalHilbertAlgebra}\leanok
  We already know that our algebra is also a co-algebra using Proposition \ref{Coalgebra.ofFiniteDimensionalHilbertAlgebra} with co-multiplication $m^*$ and co-unit $\eta^*$.
  So then it suffices to show $(m\otimes\,\id)(\id\otimes\,m^*)=m^*m$, as we get the other equality by taking adjoints.

  Let $x,y,z,w\in{B}$. Let $m^*(y)=\sum_i\alpha_i\otimes\beta_i$ for some tuples $(\alpha_i),(\beta_i)$ in $B$. Then we compute,
  \begin{align*}
   \ip{(m\otimes\id)(\id\otimes\,m^*)(x\otimes y)}{z\otimes w}
   &= \sum_i\ip{(m\otimes\id)(x\otimes\alpha_i\otimes\beta_i)}{z\otimes w}\\
   &= \sum_i\ip{x\alpha_i\otimes\beta_i}{z\otimes w}
   = \sum_i\ip{x\alpha_i}{z}\ip{\beta_i}{w}\\
   = \sum_i\ip{\alpha_i}{x^*z}\ip{\beta_i}{w}\\
   &= \sum_i\ip{\alpha_i\otimes\beta_i}{x^*z\otimes w}=\ip{m^*(y)}{x^*z\otimes w}\\
   &= \ip{y}{x^*zw}=\ip{xy}{zw}\\
   &= \ip{m^*m(x\otimes y)}{z\otimes w}.
  \end{align*}
  Thus $(m\otimes\id)(\id\otimes\,m^*)=m^*m$.
 \end{proof}
