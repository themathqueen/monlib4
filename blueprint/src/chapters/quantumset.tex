\chapter{Quantum sets}

 \begin{definition}\label{QuantumSet}
  \uses{LinearMap.real, FiniteDimensionalCoAlgebra_isFrobeniusAlgebra_of, isAlgHom_iff_adjoint_isCoalgHom}
  \lean{QuantumSet}\leanok
  Given a $^*$-algebra $(\mathcal{A},m,\eta)$ that is also a finite-dimensional Hilbert space,
  we say it is a \textit{quantum set} when there is a modular automorphism $\sigma_t\colon\mathcal{A}\cong\mathcal{A}$, which is an algebra automorphism for each $t\in\RR$, and that the following properties are satisfied:
  \begin{enumerate}
   \item $\sigma_t\circ\sigma_s=\sigma_{t+s}$,
   \item $\sigma_0=\id$,
   \item $\sigma_t^{\operatorname{r}}=\sigma_{-t}$,
   \item $\sigma_t$ is self-adjoint,
   \item $\forall{x,y,z}\in\mathcal{A}:\ip{xy}{z}=\ip{y}{x^*z}$,
   \item $\forall{x,y,z}\in\mathcal{A}:\ip{xy}{z}=\ip{x}{z\sigma_{-1}(y^*)}$.
  \end{enumerate}
  By Proposition \ref{FiniteDimensionalCoAlgebra_isFrobeniusAlgebra_of}, we see that $(\mathcal{A},m,\eta)$ is a Frobenius algebra.

  And by Proposition \ref{isAlgHom_iff_adjoint_isCoalgHom}, we can see that for any $t\in\RR$, we get $\sigma_t$ is also a co-algebra homomorphism.

  We can easily see that we get $\ip{x}{y}=\eta^*(x^*y)$ for all $x,y\in\mathcal{A}$.
 \end{definition}

 Clearly, the Hilbert space of $\bigoplus_iM_{n_i}$ given by a positive and faithful linear functional $\psi$ (Definition \ref{PiMat.InnerProductSpace}) is a quantum set with modular automorphism defined in Definition \ref{modAut}.
