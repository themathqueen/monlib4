\chapter{Quantum sets}

 \begin{definition}\label{QuantumSet}
  \uses{LinearMap.real, FiniteDimensionalCoAlgebra_isFrobeniusAlgebra_of, isAlgHom_iff_adjoint_isCoalgHom}
  \lean{QuantumSet}\leanok
  Given a $^*$-algebra $(\mathcal{A},m,\eta)$ that is also a finite-dimensional Hilbert space,
  we say it is a \textit{quantum set} when there is a modular automorphism $\sigma_t\colon\mathcal{A}\cong\mathcal{A}$, which is an algebra automorphism for each $t\in\mathbb{R}$, and that the following properties are satisfied (for a fixed $k\in\mathbb{R}$):
  \begin{enumerate}
   \item $\sigma_t\circ\sigma_s=\sigma_{t+s}$,
   \item $\sigma_0=\id$,
   \item $\sigma_t^{\operatorname{r}}=\sigma_{-t}$,
   \item $\sigma_t$ is self-adjoint,
   \item $\forall{x,y,z}\in\mathcal{A}:\ip{xy}{z}=\ip{y}{\sigma_{-k}(x)^*z}$,
   \item $\forall{x,y,z}\in\mathcal{A}:\ip{xy}{z}=\ip{x}{z\sigma_{-k-1}(y^*)}$.
  \end{enumerate}
  By Proposition \ref{FiniteDimensionalCoAlgebra_isFrobeniusAlgebra_of}, we see that $(\mathcal{A},m,\eta,k)$ is a Frobenius algebra.

  And by Proposition \ref{isAlgHom_iff_adjoint_isCoalgHom}, we can see that for any $t\in\mathbb{R}$, we get $\sigma_t$ is also a co-algebra homomorphism.

  We can easily see that we get $\ip{x}{y}=\eta^*(x^*\sigma_k(y))$ for all $x,y\in\mathcal{A}$.
 \end{definition}
 \begin{remark}
  In the Lean code, we start with a $^*$-algebra and define a modular automorphism $\sigma_t$ that satisfies properties 1--3 above.
  Then we add the Hilbert space structure on top of that.
  The reason for this is so that we can define more quantum sets on the same space (with a different $k$-value; in practice, we either use $k=0$ for the GNS-inner product, or $k=-1/2$ for the KMS-inner product).
  This way minimises the instance diamonds created.

  Normally, we start with a finite-dimensional $^*$-algebra $(\mathcal{A},m,\eta)$ with a faithful and positive linear functional $\psi$. Then define the Hilbert space structure by $\ip{x}{y}=\psi(x^*y)$.
  The modular automorphism is then defined by $\sigma_t(x)=Q^{-t}xQ^t$, where $Q$ is the ``weight'' of $\psi$ (in other words, $\psi(x)=\Tr(Q x)$).
  Then one has a quantum set.
 \end{remark}

 Clearly, the Hilbert space of $\bigoplus_iM_{n_i}$ given by a positive and faithful linear functional $\psi$ (Definition \ref{PiMat.InnerProductSpace}) is a quantum set with modular automorphism defined in Definition \ref{modAut}.
 
 \begin{lemma}\label{ket_real}\uses{LinearMap.real, ket}\lean{ket_real}\leanok
  Given a Hilbert space $\Hcal$ that is also a $^*$-algebra, and $x\in\Hcal$, we get $\ket{x}^{\operatorname{r}}=\ket{x^*}$.
 \end{lemma}
 \begin{proof}\leanok
  \[{\ket{x}(1^*)}^*=\ket{x^*}(1).\]
 \end{proof}

 \begin{lemma}\label{bra_real}\uses{LinearMap.real, bra, QuantumSet}\lean{bra_real}\leanok
  Given a quantum set $(\mathcal{A},m,\eta,\sigma_r)$ and $x\in\mathcal{A}$, we get $\bra{x}^{\operatorname{r}}=\bra{\sigma_{-1}(x^*)}$.
 \end{lemma}
 \begin{proof}\leanok
  \[{\bra{x}(y^*)}^*=\ip{y^*}{x}=\ip{\sigma_{-1}(x^*)}{y}=\bra{\sigma_{-1}(x^*)}(y).\]
 \end{proof}
 \begin{corollary}\label{rankOne_real}
  \uses{QuantumSet, LinearMap.real}
  \lean{rankOne_real}\leanok
  Given quantum sets $(\mathcal{A}_1,m_1,\eta_1,\sigma_r)$ and $(\mathcal{A}_2,m_2,\eta_2,\vartheta_r)$, and elements $a\in\mathcal{A}_1$ and $b\in\mathcal{A}_2$, we get
  \[\ketbra{a}{b}^{\operatorname{r}}=\ketbra{a^*}{\vartheta_{-1}(b^*)}.\]
 \end{corollary}
 \begin{proof}\uses{ket_real, bra_real, LinearMap.real_comp}\leanok
  Use Lemmas \ref{ket_real}, \ref{bra_real}, and \ref{LinearMap.real_comp}.
 \end{proof}

 \begin{proposition}\label{LinearMap.adjoint_real_eq}
  \uses{LinearMap.real, QuantumSet}
  \lean{LinearMap.adjoint_real_eq}\leanok
  Given quantum sets $(\mathcal{A}_1,m_1,\eta_1,\sigma_r)$ and $(\mathcal{A}_2,m_2,\eta_2,\vartheta_r)$, and $f\colon\mathcal{A}_1\to\mathcal{A}_2$, we get
  \[ f^{*\operatorname{r}}=\vartheta_1f^{\operatorname{r}*}\sigma_{-1}.\]
 \end{proposition}
 \begin{proof}\leanok
  Let $x\in{\mathcal{A}_2}$ and $y\in \mathcal{A}_1$, and compute,
  \begin{align*}
   \ip{{A^{*\operatorname{r}}}(x)}{y} &= \ip{{A^*(x^*)}^*}{y} =\ip{y^*}{\sigma_{-1}(A^*(x^*))}\\
   &= \ip{A(\sigma_{-1}(y^*))}{x^*}\\
   &= \ip{A({\sigma_{1}(y)}^*)}{x^*}
   = \ip{x}{\vartheta_{-1}A^{\operatorname{r}}\sigma_{1}(y)}\\
   &= \ip{\sigma_{1}A^{\operatorname{r}*}\vartheta_{-1}(x)}{y}
  \end{align*}
  Thus, $A^{*\operatorname{r}}=\sigma_{1}A^{\operatorname{r}*}\vartheta_{-1}$.
 \end{proof}

 \begin{proposition}\label{modAut_one_eq_left_twist}
  \uses{QuantumSet}
  \lean{QuantumSet.rTensor_counit_mul_comp_comm_comp_rTensor_comul_unit_eq_modAut_one}\leanok
  Given a quantum set $(\mathcal{A},m,\eta,\sigma_r)$, we get
  \[(\eta^*m\otimes\id)(\id\otimes\,\varkappa)(m^*\eta\otimes\id)=\sigma_1.\]
  Here, $\varkappa$ is the identification $\mathcal{A}\otimes\mathcal{A}\cong\mathcal{A}\otimes\mathcal{A}$ given by $x\otimes y\mapsto y\otimes x$.
 \end{proposition}
 \begin{proof}\leanok
  Let $x,y\in\mathcal{A}$, and let $m^*(1)=\sum_i\alpha_i\otimes\beta_i$ for some tuples $(\alpha_i),(\beta_i)$ in $\mathcal{A}$.
  Then we compute,
  \begin{align*}
   &\,\ip{(\eta^*m\otimes\id)(\id\otimes\,\varkappa)(m^*\eta\otimes\id)(x)}{y}\\
   =& \sum_i\ip{(\eta^*m\otimes\id)(\id\otimes\,\varkappa)(\alpha_i\otimes\beta_i\otimes x)}{y}\\
   =&\, \sum_i\ip{(\eta^*m\otimes\id)(\alpha_i\otimes x\otimes \beta_i)}{y}\\
   =&\, \sum_i\ip{\eta^*(\alpha_ix)\beta_i}{y}
   = \sum_i\ip{\alpha_ix}{1}\ip{\beta_i}{y}\\
   =&\, \sum_i\ip{\alpha_i}{\sigma_{-1}(x^*)}\ip{\beta_i}{y}\\
   =&\, \sum_i\ip{\alpha_i\otimes\beta_i}{\sigma_{-1}(x^*)\otimes y}=\ip{m^*(1)}{\sigma_{-1}(x^*)\otimes y}\\
   =&\, \ip{1}{\sigma_{-1}(x^*)y}=\ip{\sigma_{1}(x)}{y}.
  \end{align*}
 \end{proof}


 \begin{proposition}\label{modAut_neg_one_eq_right_twist}
  \uses{QuantumSet}
  \lean{QuantumSet.lTensor_counit_mul_comp_comm_comp_lTensor_comul_unit_eq_modAut_neg_one}\leanok
  Given a quantum set $(\mathcal{A},m,\eta,\sigma_r)$, we get
  \[(\id\otimes\,\eta^*m)(\varkappa\otimes\id)(\id\otimes\,m^*\eta)=\sigma_{-1}.\]
  Here, $\varkappa$ is the identification $\mathcal{A}\otimes\mathcal{A}\cong\mathcal{A}\otimes\mathcal{A}$ given by $x\otimes y\mapsto y\otimes x$.
 \end{proposition}
 \begin{proof}\leanok
  We let $x,y\in\mathcal{A}$ and $m^*(1)=\sum_i\alpha_i\otimes\beta_i$ for some tuples $(\alpha_i),(\beta_i)$ in $\mathcal{A}$. Then we compute,
  \begin{align*}
   & \ip{(\id\otimes\,\eta^*m)(\varkappa\otimes\id)(\id\otimes\,m^*\eta)(x)}{y}\\
   =& \sum_i\ip{(\id\otimes\,\eta^*m)(\varkappa\otimes\id)(x\otimes\alpha_i\otimes\beta_i)}{y}\\
   =& \sum_i\ip{(\id\otimes\,\eta^*m)(\alpha_i\otimes x\otimes\beta_i)}{y}\\
   =& \sum_i\ip{\ip{1}{x\beta_i}\alpha_i}{y} = \sum_i\ip{\alpha_i}{y}\ip{x\beta_i}{1}\\
   =& \sum_i\ip{\alpha_i}{y}\ip{\beta_i}{x^*}\\
   =& \ip{m^*(1)}{y\otimes x^*}=\ip{1}{yx^*}\\
   =& \ip{\sigma_{-1}(x)}{y}.
  \end{align*}
 \end{proof}

 \begin{lemma}\label{counit_mul_modAut_symm'}
  \uses{QuantumSet}
  \lean{counit_mul_modAut_symm'}\leanok
  Given a quantum set $(\mathcal{A},m,\eta,\sigma)$, we get
  \[\eta^*(a\sigma_r(b)) = \eta^*(\sigma_{r + 1}(b)a),\]
  for any $r\in\mathbb{R}$.
 \end{lemma}
 \begin{proof}\leanok
  \begin{align*}
   \eta^*(a\sigma_r(b)) &= \ip{1}{a\sigma_r(b)} = \ip{\sigma_{-(r+1)}(b^*)}{a}\\
   &= \ip{{\sigma_{r+1}(b)}^*}{a}=\eta^*(\sigma_{r+1}(b)a).
  \end{align*}
 \end{proof}

 \begin{corollary}\label{moving_twists}
  \uses{counit_mul_modAut_symm'}
  \lean{counit_comp_mul_comp_rTensor_modAut}\leanok
  Given a quantum set $(\mathcal{A},m,\eta,\sigma)$, we get
  \[\eta^* m(\sigma_{1}\otimes\id)=\eta^*\circ m\circ\varkappa.\]
 \end{corollary}
 \begin{proof}\leanok
  Use Lemma \ref{counit_mul_modAut_symm'}.
 \end{proof}
 \begin{corollary}\label{moving_twists_2}
  \uses{counit_mul_modAut_symm'}
  \lean{counit_comp_mul_comp_lTensor_modAut}\leanok
  Given a quantum set $(\mathcal{A},m,\eta,\sigma)$, we get
  \[\eta^* m(\id\otimes\sigma_{-1})=\eta^*\circ m\circ\varkappa.\]
 \end{corollary}
 \begin{proof}\leanok
  Use Lemma \ref{counit_mul_modAut_symm'}.
 \end{proof}
 
 % =\unitadjoint\mul(\id\otimes\,\ext{-1})$.
 \begin{definition}\label{QuantumSet.Psi}
  \uses{QuantumSet}
  \lean{QuantumSet.Psi}\leanok
  Given quantum sets $(\mathcal{A}_1,m_1,\eta_1,\sigma_r)$ and $(\mathcal{A}_2,m_2,\eta_2,\vartheta_r)$, then for each $t,r\in\mathbb{R}$, we define $\Psi_{t,r}$ to be the linear isomorphism from $\Bcal(\mathcal{A}_1,\mathcal{A}_2)$ to $\mathcal{A}_2\otimes\mathcal{A}_1^{\operatorname{op}}$ given by
  \[\Psi_{t,r}(\ketbra{a}{b})=\vartheta_t(a)\otimes{\sigma_r(b)}^{*\operatorname{op}},\]
  with inverse given by
  \[\Psi_{t,r}^{-1}(a\otimes b^{\operatorname{op}})=\ketbra{\sigma_{-t}(a)}{\vartheta_{-r}(b^*)}.\]
 \end{definition}