\chapter{Ket-bra operators}
 Let $E_1,E_2,E_3$ be inner product spaces over $\CC$.

 \begin{definition}\label{rankOne}\lean{rankOne}\leanok
  A \textit{ket-bra} operator $\ketbra{\cdot}{\cdot}$ is defined as the linear map from $E_2$ to the anti-linear map $E_1\to\Bcal(E_1,E_2)$ and is given by
  \begin{align*}
   x\mapsto(y\mapsto(u\mapsto\ip{y}{u}x)).
  \end{align*}
 \end{definition} 
 
 So given $x\in{E_2}$ and $y\in{E_1}$, we write $\ketbra{x}{y}$ to mean the map $u\mapsto\ip{y}{u}x$.

 \begin{lemma}\uses{rankOne}\label{linearMap_comp_rankOne}\lean{ContinuousLinearMap.comp_rankOne}\leanok
  Given a linear map $T_1\in\Bcal(E_2,E_3)$ and elements $x\in E_2$, $y\in{E_1}$, we get, $T_1\circ\ketbra{x}{y}=\ketbra{T_1(x)}{y}$
 \end{lemma}
 \begin{proof}\leanok
  Straightforward computation.
 \end{proof}
 
 \begin{lemma}\uses{rankOne}\label{rankOne_comp_linearMap}\lean{ContinuousLinearMap.rankOne_comp}\leanok
  Given a linear map $T_2\in\Bcal(E_3,E_1)$ and elements $x\in E_2$, $y\in E_1$, we get $\ketbra{x}{y}\circ T_2=\ketbra{x}{T_2^*(y)}$.
 \end{lemma}
 \begin{proof}\leanok
  Straightforward computation.
 \end{proof}
 
 \begin{lemma}\uses{rankOne}\label{rankOne_adjoint}\lean{rankOne_adjoint}\leanok
  Given $x\in E_2$ and $y\in E_1$, we get $\ketbra{x}{y}^*=\ketbra{y}{x}$.
 \end{lemma}
 \begin{proof}\leanok
  Straightforward computation.
 \end{proof}

 \begin{lemma}\uses{rankOne}\label{sum_rankOne_onb_eq_id}\lean{rankOne.sum_orthonormalBasis_eq_id}\leanok
  Given an orthonormal basis $(u_i)$ of a $\CC$-inner product space $E$, we get $\sum_i\ketbra{u_i}{u_i}=\id$.
 \end{lemma}
 \begin{proof}\leanok
  Straightforward computation.
 \end{proof}

 \begin{lemma}\label{continuous_linear_map.centralizer}\lean{ContinuousLinearMap.commutes_with_all_iff}\leanok
  Given a Hilbert space $\Hcal$, we have ${\Bcal(\Hcal)}^\prime=\{\alpha\id:\alpha\in\CC\}$.\\
  In other words, $x\in\Bcal(\Hcal)$ commutes with all operators $y\in\Bcal(\Hcal)$ if and only if $x=\alpha\id$ for some $\alpha\in\CC$.
 \end{lemma}
 \begin{proof}\uses{rankOne}\uses{rankOne_adjoint}\uses{rankOne_comp_linearMap}\uses{linearMap_comp_rankOne}\leanok
  Let $x\in\Bcal(\Hcal)$. Obviously, if $x=\alpha\id$ for some $\alpha\in\CC$, then it commutes with every other operator. Now suppose $x$ commutes with every operator in $\Bcal(\Hcal)$.
  So this means $\ketbra{a}{x^*(b)}=\ketbra{a}{b}x=x\ketbra{a}{b}=\ketbra{x(a)}{b}$ for all $a,b\in\Hcal$.
  Suppose there exists some non-zero $a\in\Hcal$, otherwise this is trivial. Then, for any $b\in\Hcal$, we have
  \[x(b)=\frac{\norm{a}^2}{\norm{a}^2}x(b)=\frac{1}{\norm{a}^2}\ketbra{x(b)}{a}(a)=\frac{1}{\norm{a}^2}\ketbra{b}{x^*(a)}(a)=\frac{\ip{x^*(a)}{a}}{\norm{a}^2}b.\]
  Thus $x=\alpha\id$ where $\alpha=\ip{x^*(a)}{a}/\norm{a}^2$.
 \end{proof}

 \begin{proposition}\uses{rankOne}\label{colinear_of_rankOne_self_eq_rankOne_self}\lean{colinear_of_rankOne_self_eq_rankOne_self}\leanok
  Let $\Hcal$ be a Hilbert space, and let $x,y\in\Hcal$. Then if $\ketbra{x}{x}=\ketbra{y}{y}$, then there exists some $0\neq\alpha\in\CC$ such that $x=\alpha{y}$ (i.e., they are co-linear).
 \end{proposition}
 \begin{proof}\uses{rankOne}\leanok
  Suppose $\ketbra{x}{x}=\ketbra{y}{y}$. Then it is clear that we get $x=0$ if and only if $y=0$. So we assume $x\neq0$ (and so $y\neq0$), otherwise this is trivial.
  Then we have \[\norm{x}^2x=\ketbra{x}{x}(x)=\ketbra{y}{y}(x)=\ip{y}{x}y.\]
  And as $x\neq0$, we get $x=\dfrac{\ip{y}{x}}{\norm{x}^2}y$. We have $\ip{y}{x}\neq0$ (otherwise, this would mean $x=0$ which is a contradiction). Thus we can let $\alpha=\ip{y}{x}/\norm{x}^2\neq0$ such that $x=\alpha{y}$.
 \end{proof}

\chapter{The inner product}

 \begin{definition}\label{linear_functional.matrix}\lean{Module.Dual.matrix}\leanok
  Given a linear functional $\phi\colon M_n \to \CC$, we define $\phi_Q=\sum_{i,j}\phi(e_{ij})e_{ji}$.
 \end{definition}

 \begin{lemma}\label{linear_functional_eq_trace}\uses{linear_functional.matrix}\lean{Module.Dual.apply}\leanok
  Given a linear functional $\phi\colon M_n\to\CC$, we get $\phi$ is given by $x\mapsto\Tr(\phi_Q \,x)$.
 \end{lemma}
 \begin{proof}\leanok
  Straightforward computation.
 \end{proof}

 \begin{definition}\label{LinearMap.IsPosMap}\lean{LinearMap.IsPosMap}\leanok
  Given C$^*$-algebras $A,C$, we say a linear map $f\colon A \to C$ is \textit{positive} when $0\leq f(a)$ for all $0\leq a$. In other words, $f$ maps positive elements in $A$ to positive elements in $C$.
 \end{definition}

 


 